\documentclass{article}
\usepackage{graphicx} % Required for inserting images
\usepackage[T1]{fontenc}

\renewcommand*\contentsname{Spis treści}

\title{Kasa samoobsługowa - dokumentacja}
\author{Marcin Wilk, Dominik Gorgosch, Norbert Jaśniewicz}
\date{marzec 2023}

\begin{document}

\maketitle
\tableofcontents
\newpage

\section{Wymagania}
    \subsection{Wstęp}
         Kasa samoobsługowa pozwala klientowi samodzielnie skanować i płacić za produkty w sklepie. Systemy takie są rozmieszczane zazwyczaj w sklepach sieci handlowych. Zasadniczo taki system posiada trzy rodzaje użytkowników:
            \begin{itemize}
                \item Klient - osoba kupująca produkty w danym sklepie.
                \item Obsługa sklepu (pracownik sklepu) - asystuje klientowi w razie problemów z kasą.
                \item Obsługa techniczna (serwisant) - pracownicy sklepu lub podwykonawcy odpowiedzialni za serwisowanie urządzenia w razie awarii.
            \end{itemize}
    \subsection{Podstawowe wymagania}
    System musi umożliwiać:
    \begin{itemize}
        \item Skanowanie produktów.
        \item Ważenie produktów.
        \item Obliczanie kosztu zakupów.
        \item Płatność za produkty.
        \item Utrudnienie kradzieży/oszustwa.
        \item Dostęp do interfejsu pracownika sklepu jedynie upoważnionym osobom.
        \item Dostęp do interfejsu serwisanta jedynie upoważnionym osobom.
        \item Kontrolowanie sprzedaży produktów z ograniczeniami wiekowymi.
    \end{itemize}
\section{Przypadki użycia}
\begin{center}
\end{center}
\subsection{Skanowanie produktów}
\begin{enumerate}
\item \textbf{Nazwa przypadku użycia}: Skanowanie produktów
\item \textbf{Priorytet}: Wysoki
\item \textbf{Aktor podstawowy}: Użytkownik
\item \textbf{Typ opisu}: Mało detali
\item \textbf{Udziałowcy i cele}:
\begin{itemize}
\item Użytkownik: Chce kupić produkty i zapłacić za nie.
\end{itemize}
\item \textbf{Typ wyzwalacza}: Użytkownik zaczyna skanowanie produktów.
\item \textbf{Powiązania}: Przypadek użycia jest powiązany z wprowadzaniem kodów rabatowych.
\item \textbf{Asocjacja}: Brak asocjacji z innymi przypadkami użycia.
\item \textbf{Zawieranie}: Brak zawierania.
\item \textbf{Rozszerzenie}: Brak rozszerzenia.
\item \textbf{Generalizacja}: Brak generalizacji.
\item \textbf{Zwykły przepływ zdarzeń}:
\begin{enumerate}
\item Użytkownik zaczyna skanowanie produktów.
\item Kasa samoobsługowa skanuje produkty i wyświetla ich ceny.
\item Kasa samoobsługowa aktualizuje cenę koszyka.
\item Użytkownik odkłada produkt do strefy pakowania.
\item Użytkownik kontynuuje skanowanie produktów lub przechodzi do wprowadzenia kodów rabatowych lub płatności.
\end{enumerate}
\item \textbf{Przepływy poboczne}:
\begin{enumerate}
\item Produkt wymaga zatwierdzenia wiekowego.
\item Uruchamia się przypadek użycia zatwierdzenia wieku przez pracownika.
\end{enumerate}
\item \textbf{Przepływy alternatywne}:
\begin{enumerate}
\item Waga produktów w strefie pakowania nie zgadza się.
\item Uruchamia się przypadek użycia sprawdzenia wagi przez pracownika.
\end{enumerate}
\end{enumerate}
\subsection{Ważenie produktu}
\begin{enumerate}
\item \textbf{Nazwa przypadku użycia}: Ważenie Produktu
\item \textbf{Priorytet}: Średni
\item \textbf{Aktor podstawowy}: Użytkownik
\item \textbf{Typ opisu}: Mało detali
\item \textbf{Udziałowcy i cele}:
\begin{itemize}
\item Użytkownik: Chce zważyć produkt, którego waga nie została określona.
\end{itemize}
\item \textbf{Typ wyzwalacza}: Użytkownik wybiera opcję ważenia produktu na ekranie kasy samoobsługowej.
\item \textbf{Powiązania}: Przypadek użycia jest powiązany z skanowaniem produktów, ponieważ może być wywołany w trakcie skanowania produktów, gdy waga produktu nie została określona.
\item \textbf{Zawieranie}: Brak zawierania.
\item \textbf{Asocjacja}: Brak asocjacji z innymi przypadkami użycia.
\item \textbf{Rozszerzenie}: Brak rozszerzenia.
\item \textbf{Generalizacja}: Brak generalizacji.
\item \textbf{Zwykły przepływ zdarzeń}:
\begin{enumerate}
\item Użytkownik wybiera opcję ważenia produktu na ekranie kasy samoobsługowej.
\item Kasa samoobsługowa wyświetla komunikat proszący użytkownika o umieszczenie produktu na wadze.
\item Użytkownik umieszcza produkt na wadze.
\item Kasa samoobsługowa wyświetla wagę produktu na ekranie.
\item Kasa samoobsługowa aktualizuje cenę koszyka.
\item Użytkownik odkłada produkt do strefy pakowania.
\item Użytkownik kontynuuje skanowanie produktów.
\end{enumerate}
\item \textbf{Przepływy poboczne}:
\begin{enumerate}
\item Waga produktu nie jest prawidłowa.
\item Kasa samoobsługowa wyświetla komunikat o błędnej wadze i prosi użytkownika o ponowne umieszczenie produktu na wadze.
\end{enumerate}
\item \textbf{Przepływy alternatywne}:
\begin{enumerate}
\item Waga produktów w strefie pakowania nie zgadza się.
\item Uruchamia się przypadek użycia sprawdzenia wagi przez pracownika.
\end{enumerate}
\end{enumerate}
\subsection{Wprowadź kod rabatowy}
\begin{enumerate}
\item \textbf{Priorytet}: Średni
\item \textbf{Aktor podstawowy}: Użytkownik
\item \textbf{Typ opisu}: Mało detali
\item \textbf{Udziałowcy i cele}:
\begin{itemize}
\item Użytkownik: Chce zastosować kod rabatowy do swojego zamówienia.
\end{itemize}
\item \textbf{Typ wyzwalacza}: Użytkownik wprowadza kod rabatowy.
\item \textbf{Powiązania}: Przypadek użycia jest powiązany z zeskanowaniem produktów i wyborem sposobu płatności.
\item \textbf{Asocjacja}: Brak asocjacji z innymi przypadkami użycia.
\item \textbf{Zawieranie}: Brak zawierania.
\item \textbf{Rozszerzenie}: Brak rozszerzenia.
\item \textbf{Generalizacja}: Brak generalizacji.
\item \textbf{Zwykły przepływ zdarzeń}:
\begin{enumerate}
\item Użytkownik wybiera opcję wprowadzenia kodu rabatowego na ekranie kasy samoobsługowej.
\item Użytkownik wprowadza kod rabatowy.
\item Kasa samoobsługowa sprawdza poprawność kodu rabatowego i wyświetla nowe ceny produktów.
\item Użytkownik kontynuuje skanowanie produktów lub przechodzi do wprowadzenia płatności.
\end{enumerate}
\item \textbf{Przepływy poboczne}:
\begin{enumerate}
\item Użytkownik wprowadza niepoprawny kod rabatowy.
\item Kasa samoobsługowa wyświetla komunikat o błędnym kodzie rabatowym i prosi użytkownika o wprowadzenie poprawnego kodu.
\end{enumerate}
\item \textbf{Przepływy alternatywne}: Brak przepływów alternatywnych.
\end{enumerate}
\subsection{Dokonaj płatności}
\begin{enumerate}
\item \textbf{Nazwa przypadku użycia}: Dokonaj płatności
\item \textbf{Priorytet}: Wysoki
\item \textbf{Aktor podstawowy}: Użytkownik
\item \textbf{Typ opisu}: Mało detali
\item \textbf{Udziałowcy i cele}:
\begin{itemize}
\item Użytkownik: Chce zapłacić za swoje produkty i zakończyć transakcję.
\end{itemize}
\item \textbf{Typ wyzwalacza}: Użytkownik wybiera opcję płatności na ekranie kasy samoobsługowej.
\item \textbf{Powiązania}: Przypadek użycia jest powiązany z zeskanowaniem produktów i wprowadzeniem kodów rabatowych.
\item \textbf{Asocjacja}: Brak asocjacji z innymi przypadkami użycia.
\item \textbf{Zawieranie}: Brak zawierania.
\item \textbf{Rozszerzenie}: Brak rozszerzenia.
\item \textbf{Generalizacja}: Brak generalizacji.
\item \textbf{Zwykły przepływ zdarzeń}:
\begin{enumerate}
\item Użytkownik wybiera opcję płatności na ekranie kasy samoobsługowej.
\item Użytkownik wprowadza metodę płatności, np. kartę kredytową lub gotówkę.
\item Kasa samoobsługowa przetwarza płatność i wydaje paragon.
\item Użytkownik odbiera paragon i kończy transakcję.
\end{enumerate}
\item \textbf{Przepływy poboczne}:
Brak przepływów pobocznych
\item \textbf{Przepływy alternatywne}:
\begin{enumerate}
    \item Brak środków na koncie 
    \item Urządzenie wyświetla komunikat o braku środków na koncie
\end{enumerate}
\end{enumerate}
\subsection{Zatwierdzenie wieku}
\begin{enumerate}
\item \textbf{Priorytet}: Wysoki
\item \textbf{Aktor podstawowy}: Pracownik
\item \textbf{Typ opisu}: Dużo
\item \textbf{Udziałowcy i cele}:
\begin{itemize}
\item \textbf{Pracownik}: potwierdzenie, że klient ma wystarczający wiek, aby zakupić produkt o ograniczeniach wiekowych.
\item \textbf{Klient}: zakup produktu o ograniczeniach wiekowych.
\end{itemize}
\item \textbf{Typ wyzwalacza}: Pracownik zauważa, że klient chce kupić produkt o ograniczeniach wiekowych.
\item \textbf{Powiązania}: Brak powiązań
\item \textbf{Asocjacja}: Związany jest z przypadkami użycia skanowanie oraz ważenie produktów, ponieważ bez zatwierdzenia wieku przez pracownika klient nie może zakupić produktu o ograniczeniach wiekowych.
\item \textbf{Zawieranie}: Brak zawierania.
\item \textbf{Rozszerzenie}: Brak
\item \textbf{Generalizacja}: Brak
\item \textbf{Zwykły przepływ zdarzeń}:
\begin{enumerate}
\item Klient składa zamówienie i skanuje produkt o ograniczeniach wiekowych.
\item Kasa samoobsługowa wyświetla komunikat o konieczności zatwierdzenia wieku przez pracownika.
\item Pracownik odblokowuje kiosk, wpisując swoje dane uwierzytelniające lub skanując swój identyfikator pracownika.
\item Kasa samoobsługowa wyświetla informację o ograniczeniach wiekowych dla produktu oraz pyta pracownika, czy klient ma wystarczający wiek, aby zakupić ten produkt.
\item Pracownik potwierdza, że klient ma wystarczający wiek i wpisuje swój kod lub skanuje swój identyfikator pracownika.
\item Kasa samoobsługowa zatwierdza zakup.
\end{enumerate}
\item \textbf{Przepływy poboczne}:
\begin{itemize}
\item Pracownik potwierdza, że klient nie ma wystarczającego wieku, aby zakupić produkt. Kasa samoobsługowa anuluje transakcję i wyświetla komunikat o niedozwolonym zakupie.
\end{itemize}
\end{enumerate}
\subsection{Zatwierdzenie nieprawidłowej wagi}
\begin{enumerate}
\item \textbf{Priorytet}: wysoki
\item \textbf{Aktor podstawowy}: pracownik
\item \textbf{Typ opisu}: dużo detali
\item \textbf{Udziałowcy i cele}: Pracownik - zatwierdzenie nieprawidłowej wagi produktu.
\item \textbf{Typ wyzwalacza}: Pracownik otrzymuje komunikat o nieprawidłowej wadze produktu z systemu samoobsługowego.
\item \textbf{Powiązania}: Ten przypadek użycia jest powiązany z przypadkiem użycia Skanowania Produktu i Ważenie Produktu.
\item \textbf{Asocjacja}: Ten przypadek użycia jest połączony z przypadkiem użycia Skanowania Produktu i Ważenie Produktu.
\item \textbf{Zawieranie}: Brak zawierania.
\item \textbf{Rozszerzenie}: Brak rozszerzeń
\item \textbf{Generalizacja}: Nie ma generalizacji dla tego przypadku użycia.
\item \textbf{Zwykły przepływ zdarzeń}:
\begin{enumerate}
\item Pracownik strefy pakowania otrzymuje komunikat o nieprawidłowej wadze produktu z systemu samoobsługowego.
\item Pracownik strefy pakowania weryfikuje wagę produktu i porównuje ją z wagą wyświetloną na kasie samoobsługowej.
\item Jeśli waga produktu jest nieprawidłowa, pracownik strefy pakowania wprowadza właściwą wagę produktu do systemu samoobsługowego.
\item System samoobsługowy zatwierdza nową wagę produktu i kontynuuje proces pakowania.
\end{enumerate}
\item \textbf{Przepływy poboczne}:
\begin{itemize}
\item Jeśli pracownik strefy pakowania nie może zweryfikować wagi produktu, to może skonsultować się z innym pracownikiem lub poprosić o zastosowanie innej metody weryfikacji wagi produktu.
\item Jeśli produkt jest uszkodzony lub niekompletny, pracownik strefy pakowania może zgłosić ten fakt do systemu samoobsługowego i przekazać go do magazynu.
\end{itemize}
\end{enumerate}
\end{document}
