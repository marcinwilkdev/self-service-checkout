\documentclass{article}
\usepackage{graphicx} % Required for inserting images
\usepackage[T1]{fontenc}

\renewcommand*\contentsname{Spis treści}

\title{Kasa samoobsługowa - dokumentacja}
\author{Marcin Wilk, Dominik Gorgosch, Norbert Jaśniewicz}
\date{marzec 2023}

\begin{document}

\maketitle
\tableofcontents
\newpage

\section{Wymagania}
    \subsection{Wstęp}
         Kasa samoobsługowa pozwala klientowi samodzielnie skanować i płacić za produkty w sklepie. Systemy takie są rozmieszczane zazwyczaj w sklepach sieci handlowych. Zasadniczo taki system posiada trzy rodzaje użytkowników:
            \begin{itemize}
                \item Klient - osoba kupująca produkty w danym sklepie.
                \item Obsługa sklepu (pracownik sklepu) - asystuje klientowi w razie problemów z kasą.
                \item Obsługa techniczna (serwisant) - pracownicy sklepu lub podwykonawcy odpowiedzialni za serwisowanie urządzenia w razie awarii.
            \end{itemize}
    \subsection{Podstawowe wymagania}
    System musi umożliwiać:
    \begin{itemize}
        \item Skanowanie produktów.
        \item Ważenie produktów.
        \item Obliczanie kosztu zakupów.
        \item Płatność za produkty.
        \item Utrudnienie kradzieży/oszustwa.
        \item Dostęp do interfejsu pracownika sklepu jedynie upoważnionym osobom.
        \item Dostęp do interfejsu serwisanta jedynie upoważnionym osobom.
        \item Kontrolowanie sprzedaży produktów z ograniczeniami wiekowymi.
    \end{itemize}
\section{Specyfikacja}

\end{document}
